\documentclass[11pt]{article}

    \usepackage[breakable]{tcolorbox}
    \usepackage{parskip} % Stop auto-indenting (to mimic markdown behaviour)
    
    \usepackage{iftex}
    \ifPDFTeX
    	\usepackage[T1]{fontenc}
    	\usepackage{mathpazo}
    \else
    	\usepackage{fontspec}
    \fi

    % Basic figure setup, for now with no caption control since it's done
    % automatically by Pandoc (which extracts ![](path) syntax from Markdown).
    \usepackage{graphicx}
    % Maintain compatibility with old templates. Remove in nbconvert 6.0
    \let\Oldincludegraphics\includegraphics
    % Ensure that by default, figures have no caption (until we provide a
    % proper Figure object with a Caption API and a way to capture that
    % in the conversion process - todo).
    \usepackage{caption}
    \DeclareCaptionFormat{nocaption}{}
    \captionsetup{format=nocaption,aboveskip=0pt,belowskip=0pt}

    \usepackage[Export]{adjustbox} % Used to constrain images to a maximum size
    \adjustboxset{max size={0.9\linewidth}{0.9\paperheight}}
    \usepackage{float}
    \floatplacement{figure}{H} % forces figures to be placed at the correct location
    \usepackage{xcolor} % Allow colors to be defined
    \usepackage{enumerate} % Needed for markdown enumerations to work
    \usepackage{geometry} % Used to adjust the document margins
    \usepackage{amsmath} % Equations
    \usepackage{amssymb} % Equations
    \usepackage{textcomp} % defines textquotesingle
    % Hack from http://tex.stackexchange.com/a/47451/13684:
    \AtBeginDocument{%
        \def\PYZsq{\textquotesingle}% Upright quotes in Pygmentized code
    }
    \usepackage{upquote} % Upright quotes for verbatim code
    \usepackage{eurosym} % defines \euro
    \usepackage[mathletters]{ucs} % Extended unicode (utf-8) support
    \usepackage{fancyvrb} % verbatim replacement that allows latex
    \usepackage{grffile} % extends the file name processing of package graphics 
                         % to support a larger range
    \makeatletter % fix for grffile with XeLaTeX
    \def\Gread@@xetex#1{%
      \IfFileExists{"\Gin@base".bb}%
      {\Gread@eps{\Gin@base.bb}}%
      {\Gread@@xetex@aux#1}%
    }
    \makeatother

    % The hyperref package gives us a pdf with properly built
    % internal navigation ('pdf bookmarks' for the table of contents,
    % internal cross-reference links, web links for URLs, etc.)
    \usepackage{hyperref}
    % The default LaTeX title has an obnoxious amount of whitespace. By default,
    % titling removes some of it. It also provides customization options.
    \usepackage{titling}
    \usepackage{longtable} % longtable support required by pandoc >1.10
    \usepackage{booktabs}  % table support for pandoc > 1.12.2
    \usepackage[inline]{enumitem} % IRkernel/repr support (it uses the enumerate* environment)
    \usepackage[normalem]{ulem} % ulem is needed to support strikethroughs (\sout)
                                % normalem makes italics be italics, not underlines
    \usepackage{mathrsfs}
    

    
    % Colors for the hyperref package
    \definecolor{urlcolor}{rgb}{0,.145,.698}
    \definecolor{linkcolor}{rgb}{.71,0.21,0.01}
    \definecolor{citecolor}{rgb}{.12,.54,.11}

    % ANSI colors
    \definecolor{ansi-black}{HTML}{3E424D}
    \definecolor{ansi-black-intense}{HTML}{282C36}
    \definecolor{ansi-red}{HTML}{E75C58}
    \definecolor{ansi-red-intense}{HTML}{B22B31}
    \definecolor{ansi-green}{HTML}{00A250}
    \definecolor{ansi-green-intense}{HTML}{007427}
    \definecolor{ansi-yellow}{HTML}{DDB62B}
    \definecolor{ansi-yellow-intense}{HTML}{B27D12}
    \definecolor{ansi-blue}{HTML}{208FFB}
    \definecolor{ansi-blue-intense}{HTML}{0065CA}
    \definecolor{ansi-magenta}{HTML}{D160C4}
    \definecolor{ansi-magenta-intense}{HTML}{A03196}
    \definecolor{ansi-cyan}{HTML}{60C6C8}
    \definecolor{ansi-cyan-intense}{HTML}{258F8F}
    \definecolor{ansi-white}{HTML}{C5C1B4}
    \definecolor{ansi-white-intense}{HTML}{A1A6B2}
    \definecolor{ansi-default-inverse-fg}{HTML}{FFFFFF}
    \definecolor{ansi-default-inverse-bg}{HTML}{000000}

    % commands and environments needed by pandoc snippets
    % extracted from the output of `pandoc -s`
    \providecommand{\tightlist}{%
      \setlength{\itemsep}{0pt}\setlength{\parskip}{0pt}}
    \DefineVerbatimEnvironment{Highlighting}{Verbatim}{commandchars=\\\{\}}
    % Add ',fontsize=\small' for more characters per line
    \newenvironment{Shaded}{}{}
    \newcommand{\KeywordTok}[1]{\textcolor[rgb]{0.00,0.44,0.13}{\textbf{{#1}}}}
    \newcommand{\DataTypeTok}[1]{\textcolor[rgb]{0.56,0.13,0.00}{{#1}}}
    \newcommand{\DecValTok}[1]{\textcolor[rgb]{0.25,0.63,0.44}{{#1}}}
    \newcommand{\BaseNTok}[1]{\textcolor[rgb]{0.25,0.63,0.44}{{#1}}}
    \newcommand{\FloatTok}[1]{\textcolor[rgb]{0.25,0.63,0.44}{{#1}}}
    \newcommand{\CharTok}[1]{\textcolor[rgb]{0.25,0.44,0.63}{{#1}}}
    \newcommand{\StringTok}[1]{\textcolor[rgb]{0.25,0.44,0.63}{{#1}}}
    \newcommand{\CommentTok}[1]{\textcolor[rgb]{0.38,0.63,0.69}{\textit{{#1}}}}
    \newcommand{\OtherTok}[1]{\textcolor[rgb]{0.00,0.44,0.13}{{#1}}}
    \newcommand{\AlertTok}[1]{\textcolor[rgb]{1.00,0.00,0.00}{\textbf{{#1}}}}
    \newcommand{\FunctionTok}[1]{\textcolor[rgb]{0.02,0.16,0.49}{{#1}}}
    \newcommand{\RegionMarkerTok}[1]{{#1}}
    \newcommand{\ErrorTok}[1]{\textcolor[rgb]{1.00,0.00,0.00}{\textbf{{#1}}}}
    \newcommand{\NormalTok}[1]{{#1}}
    
    % Additional commands for more recent versions of Pandoc
    \newcommand{\ConstantTok}[1]{\textcolor[rgb]{0.53,0.00,0.00}{{#1}}}
    \newcommand{\SpecialCharTok}[1]{\textcolor[rgb]{0.25,0.44,0.63}{{#1}}}
    \newcommand{\VerbatimStringTok}[1]{\textcolor[rgb]{0.25,0.44,0.63}{{#1}}}
    \newcommand{\SpecialStringTok}[1]{\textcolor[rgb]{0.73,0.40,0.53}{{#1}}}
    \newcommand{\ImportTok}[1]{{#1}}
    \newcommand{\DocumentationTok}[1]{\textcolor[rgb]{0.73,0.13,0.13}{\textit{{#1}}}}
    \newcommand{\AnnotationTok}[1]{\textcolor[rgb]{0.38,0.63,0.69}{\textbf{\textit{{#1}}}}}
    \newcommand{\CommentVarTok}[1]{\textcolor[rgb]{0.38,0.63,0.69}{\textbf{\textit{{#1}}}}}
    \newcommand{\VariableTok}[1]{\textcolor[rgb]{0.10,0.09,0.49}{{#1}}}
    \newcommand{\ControlFlowTok}[1]{\textcolor[rgb]{0.00,0.44,0.13}{\textbf{{#1}}}}
    \newcommand{\OperatorTok}[1]{\textcolor[rgb]{0.40,0.40,0.40}{{#1}}}
    \newcommand{\BuiltInTok}[1]{{#1}}
    \newcommand{\ExtensionTok}[1]{{#1}}
    \newcommand{\PreprocessorTok}[1]{\textcolor[rgb]{0.74,0.48,0.00}{{#1}}}
    \newcommand{\AttributeTok}[1]{\textcolor[rgb]{0.49,0.56,0.16}{{#1}}}
    \newcommand{\InformationTok}[1]{\textcolor[rgb]{0.38,0.63,0.69}{\textbf{\textit{{#1}}}}}
    \newcommand{\WarningTok}[1]{\textcolor[rgb]{0.38,0.63,0.69}{\textbf{\textit{{#1}}}}}
    
    
    % Define a nice break command that doesn't care if a line doesn't already
    % exist.
    \def\br{\hspace*{\fill} \\* }
    % Math Jax compatibility definitions
    \def\gt{>}
    \def\lt{<}
    \let\Oldtex\TeX
    \let\Oldlatex\LaTeX
    \renewcommand{\TeX}{\textrm{\Oldtex}}
    \renewcommand{\LaTeX}{\textrm{\Oldlatex}}
    % Document parameters
    % Document title
    \title{LR\_Gradient}
    
    
    
    
    
% Pygments definitions
\makeatletter
\def\PY@reset{\let\PY@it=\relax \let\PY@bf=\relax%
    \let\PY@ul=\relax \let\PY@tc=\relax%
    \let\PY@bc=\relax \let\PY@ff=\relax}
\def\PY@tok#1{\csname PY@tok@#1\endcsname}
\def\PY@toks#1+{\ifx\relax#1\empty\else%
    \PY@tok{#1}\expandafter\PY@toks\fi}
\def\PY@do#1{\PY@bc{\PY@tc{\PY@ul{%
    \PY@it{\PY@bf{\PY@ff{#1}}}}}}}
\def\PY#1#2{\PY@reset\PY@toks#1+\relax+\PY@do{#2}}

\expandafter\def\csname PY@tok@w\endcsname{\def\PY@tc##1{\textcolor[rgb]{0.73,0.73,0.73}{##1}}}
\expandafter\def\csname PY@tok@c\endcsname{\let\PY@it=\textit\def\PY@tc##1{\textcolor[rgb]{0.25,0.50,0.50}{##1}}}
\expandafter\def\csname PY@tok@cp\endcsname{\def\PY@tc##1{\textcolor[rgb]{0.74,0.48,0.00}{##1}}}
\expandafter\def\csname PY@tok@k\endcsname{\let\PY@bf=\textbf\def\PY@tc##1{\textcolor[rgb]{0.00,0.50,0.00}{##1}}}
\expandafter\def\csname PY@tok@kp\endcsname{\def\PY@tc##1{\textcolor[rgb]{0.00,0.50,0.00}{##1}}}
\expandafter\def\csname PY@tok@kt\endcsname{\def\PY@tc##1{\textcolor[rgb]{0.69,0.00,0.25}{##1}}}
\expandafter\def\csname PY@tok@o\endcsname{\def\PY@tc##1{\textcolor[rgb]{0.40,0.40,0.40}{##1}}}
\expandafter\def\csname PY@tok@ow\endcsname{\let\PY@bf=\textbf\def\PY@tc##1{\textcolor[rgb]{0.67,0.13,1.00}{##1}}}
\expandafter\def\csname PY@tok@nb\endcsname{\def\PY@tc##1{\textcolor[rgb]{0.00,0.50,0.00}{##1}}}
\expandafter\def\csname PY@tok@nf\endcsname{\def\PY@tc##1{\textcolor[rgb]{0.00,0.00,1.00}{##1}}}
\expandafter\def\csname PY@tok@nc\endcsname{\let\PY@bf=\textbf\def\PY@tc##1{\textcolor[rgb]{0.00,0.00,1.00}{##1}}}
\expandafter\def\csname PY@tok@nn\endcsname{\let\PY@bf=\textbf\def\PY@tc##1{\textcolor[rgb]{0.00,0.00,1.00}{##1}}}
\expandafter\def\csname PY@tok@ne\endcsname{\let\PY@bf=\textbf\def\PY@tc##1{\textcolor[rgb]{0.82,0.25,0.23}{##1}}}
\expandafter\def\csname PY@tok@nv\endcsname{\def\PY@tc##1{\textcolor[rgb]{0.10,0.09,0.49}{##1}}}
\expandafter\def\csname PY@tok@no\endcsname{\def\PY@tc##1{\textcolor[rgb]{0.53,0.00,0.00}{##1}}}
\expandafter\def\csname PY@tok@nl\endcsname{\def\PY@tc##1{\textcolor[rgb]{0.63,0.63,0.00}{##1}}}
\expandafter\def\csname PY@tok@ni\endcsname{\let\PY@bf=\textbf\def\PY@tc##1{\textcolor[rgb]{0.60,0.60,0.60}{##1}}}
\expandafter\def\csname PY@tok@na\endcsname{\def\PY@tc##1{\textcolor[rgb]{0.49,0.56,0.16}{##1}}}
\expandafter\def\csname PY@tok@nt\endcsname{\let\PY@bf=\textbf\def\PY@tc##1{\textcolor[rgb]{0.00,0.50,0.00}{##1}}}
\expandafter\def\csname PY@tok@nd\endcsname{\def\PY@tc##1{\textcolor[rgb]{0.67,0.13,1.00}{##1}}}
\expandafter\def\csname PY@tok@s\endcsname{\def\PY@tc##1{\textcolor[rgb]{0.73,0.13,0.13}{##1}}}
\expandafter\def\csname PY@tok@sd\endcsname{\let\PY@it=\textit\def\PY@tc##1{\textcolor[rgb]{0.73,0.13,0.13}{##1}}}
\expandafter\def\csname PY@tok@si\endcsname{\let\PY@bf=\textbf\def\PY@tc##1{\textcolor[rgb]{0.73,0.40,0.53}{##1}}}
\expandafter\def\csname PY@tok@se\endcsname{\let\PY@bf=\textbf\def\PY@tc##1{\textcolor[rgb]{0.73,0.40,0.13}{##1}}}
\expandafter\def\csname PY@tok@sr\endcsname{\def\PY@tc##1{\textcolor[rgb]{0.73,0.40,0.53}{##1}}}
\expandafter\def\csname PY@tok@ss\endcsname{\def\PY@tc##1{\textcolor[rgb]{0.10,0.09,0.49}{##1}}}
\expandafter\def\csname PY@tok@sx\endcsname{\def\PY@tc##1{\textcolor[rgb]{0.00,0.50,0.00}{##1}}}
\expandafter\def\csname PY@tok@m\endcsname{\def\PY@tc##1{\textcolor[rgb]{0.40,0.40,0.40}{##1}}}
\expandafter\def\csname PY@tok@gh\endcsname{\let\PY@bf=\textbf\def\PY@tc##1{\textcolor[rgb]{0.00,0.00,0.50}{##1}}}
\expandafter\def\csname PY@tok@gu\endcsname{\let\PY@bf=\textbf\def\PY@tc##1{\textcolor[rgb]{0.50,0.00,0.50}{##1}}}
\expandafter\def\csname PY@tok@gd\endcsname{\def\PY@tc##1{\textcolor[rgb]{0.63,0.00,0.00}{##1}}}
\expandafter\def\csname PY@tok@gi\endcsname{\def\PY@tc##1{\textcolor[rgb]{0.00,0.63,0.00}{##1}}}
\expandafter\def\csname PY@tok@gr\endcsname{\def\PY@tc##1{\textcolor[rgb]{1.00,0.00,0.00}{##1}}}
\expandafter\def\csname PY@tok@ge\endcsname{\let\PY@it=\textit}
\expandafter\def\csname PY@tok@gs\endcsname{\let\PY@bf=\textbf}
\expandafter\def\csname PY@tok@gp\endcsname{\let\PY@bf=\textbf\def\PY@tc##1{\textcolor[rgb]{0.00,0.00,0.50}{##1}}}
\expandafter\def\csname PY@tok@go\endcsname{\def\PY@tc##1{\textcolor[rgb]{0.53,0.53,0.53}{##1}}}
\expandafter\def\csname PY@tok@gt\endcsname{\def\PY@tc##1{\textcolor[rgb]{0.00,0.27,0.87}{##1}}}
\expandafter\def\csname PY@tok@err\endcsname{\def\PY@bc##1{\setlength{\fboxsep}{0pt}\fcolorbox[rgb]{1.00,0.00,0.00}{1,1,1}{\strut ##1}}}
\expandafter\def\csname PY@tok@kc\endcsname{\let\PY@bf=\textbf\def\PY@tc##1{\textcolor[rgb]{0.00,0.50,0.00}{##1}}}
\expandafter\def\csname PY@tok@kd\endcsname{\let\PY@bf=\textbf\def\PY@tc##1{\textcolor[rgb]{0.00,0.50,0.00}{##1}}}
\expandafter\def\csname PY@tok@kn\endcsname{\let\PY@bf=\textbf\def\PY@tc##1{\textcolor[rgb]{0.00,0.50,0.00}{##1}}}
\expandafter\def\csname PY@tok@kr\endcsname{\let\PY@bf=\textbf\def\PY@tc##1{\textcolor[rgb]{0.00,0.50,0.00}{##1}}}
\expandafter\def\csname PY@tok@bp\endcsname{\def\PY@tc##1{\textcolor[rgb]{0.00,0.50,0.00}{##1}}}
\expandafter\def\csname PY@tok@fm\endcsname{\def\PY@tc##1{\textcolor[rgb]{0.00,0.00,1.00}{##1}}}
\expandafter\def\csname PY@tok@vc\endcsname{\def\PY@tc##1{\textcolor[rgb]{0.10,0.09,0.49}{##1}}}
\expandafter\def\csname PY@tok@vg\endcsname{\def\PY@tc##1{\textcolor[rgb]{0.10,0.09,0.49}{##1}}}
\expandafter\def\csname PY@tok@vi\endcsname{\def\PY@tc##1{\textcolor[rgb]{0.10,0.09,0.49}{##1}}}
\expandafter\def\csname PY@tok@vm\endcsname{\def\PY@tc##1{\textcolor[rgb]{0.10,0.09,0.49}{##1}}}
\expandafter\def\csname PY@tok@sa\endcsname{\def\PY@tc##1{\textcolor[rgb]{0.73,0.13,0.13}{##1}}}
\expandafter\def\csname PY@tok@sb\endcsname{\def\PY@tc##1{\textcolor[rgb]{0.73,0.13,0.13}{##1}}}
\expandafter\def\csname PY@tok@sc\endcsname{\def\PY@tc##1{\textcolor[rgb]{0.73,0.13,0.13}{##1}}}
\expandafter\def\csname PY@tok@dl\endcsname{\def\PY@tc##1{\textcolor[rgb]{0.73,0.13,0.13}{##1}}}
\expandafter\def\csname PY@tok@s2\endcsname{\def\PY@tc##1{\textcolor[rgb]{0.73,0.13,0.13}{##1}}}
\expandafter\def\csname PY@tok@sh\endcsname{\def\PY@tc##1{\textcolor[rgb]{0.73,0.13,0.13}{##1}}}
\expandafter\def\csname PY@tok@s1\endcsname{\def\PY@tc##1{\textcolor[rgb]{0.73,0.13,0.13}{##1}}}
\expandafter\def\csname PY@tok@mb\endcsname{\def\PY@tc##1{\textcolor[rgb]{0.40,0.40,0.40}{##1}}}
\expandafter\def\csname PY@tok@mf\endcsname{\def\PY@tc##1{\textcolor[rgb]{0.40,0.40,0.40}{##1}}}
\expandafter\def\csname PY@tok@mh\endcsname{\def\PY@tc##1{\textcolor[rgb]{0.40,0.40,0.40}{##1}}}
\expandafter\def\csname PY@tok@mi\endcsname{\def\PY@tc##1{\textcolor[rgb]{0.40,0.40,0.40}{##1}}}
\expandafter\def\csname PY@tok@il\endcsname{\def\PY@tc##1{\textcolor[rgb]{0.40,0.40,0.40}{##1}}}
\expandafter\def\csname PY@tok@mo\endcsname{\def\PY@tc##1{\textcolor[rgb]{0.40,0.40,0.40}{##1}}}
\expandafter\def\csname PY@tok@ch\endcsname{\let\PY@it=\textit\def\PY@tc##1{\textcolor[rgb]{0.25,0.50,0.50}{##1}}}
\expandafter\def\csname PY@tok@cm\endcsname{\let\PY@it=\textit\def\PY@tc##1{\textcolor[rgb]{0.25,0.50,0.50}{##1}}}
\expandafter\def\csname PY@tok@cpf\endcsname{\let\PY@it=\textit\def\PY@tc##1{\textcolor[rgb]{0.25,0.50,0.50}{##1}}}
\expandafter\def\csname PY@tok@c1\endcsname{\let\PY@it=\textit\def\PY@tc##1{\textcolor[rgb]{0.25,0.50,0.50}{##1}}}
\expandafter\def\csname PY@tok@cs\endcsname{\let\PY@it=\textit\def\PY@tc##1{\textcolor[rgb]{0.25,0.50,0.50}{##1}}}

\def\PYZbs{\char`\\}
\def\PYZus{\char`\_}
\def\PYZob{\char`\{}
\def\PYZcb{\char`\}}
\def\PYZca{\char`\^}
\def\PYZam{\char`\&}
\def\PYZlt{\char`\<}
\def\PYZgt{\char`\>}
\def\PYZsh{\char`\#}
\def\PYZpc{\char`\%}
\def\PYZdl{\char`\$}
\def\PYZhy{\char`\-}
\def\PYZsq{\char`\'}
\def\PYZdq{\char`\"}
\def\PYZti{\char`\~}
% for compatibility with earlier versions
\def\PYZat{@}
\def\PYZlb{[}
\def\PYZrb{]}
\makeatother


    % For linebreaks inside Verbatim environment from package fancyvrb. 
    \makeatletter
        \newbox\Wrappedcontinuationbox 
        \newbox\Wrappedvisiblespacebox 
        \newcommand*\Wrappedvisiblespace {\textcolor{red}{\textvisiblespace}} 
        \newcommand*\Wrappedcontinuationsymbol {\textcolor{red}{\llap{\tiny$\m@th\hookrightarrow$}}} 
        \newcommand*\Wrappedcontinuationindent {3ex } 
        \newcommand*\Wrappedafterbreak {\kern\Wrappedcontinuationindent\copy\Wrappedcontinuationbox} 
        % Take advantage of the already applied Pygments mark-up to insert 
        % potential linebreaks for TeX processing. 
        %        {, <, #, %, $, ' and ": go to next line. 
        %        _, }, ^, &, >, - and ~: stay at end of broken line. 
        % Use of \textquotesingle for straight quote. 
        \newcommand*\Wrappedbreaksatspecials {% 
            \def\PYGZus{\discretionary{\char`\_}{\Wrappedafterbreak}{\char`\_}}% 
            \def\PYGZob{\discretionary{}{\Wrappedafterbreak\char`\{}{\char`\{}}% 
            \def\PYGZcb{\discretionary{\char`\}}{\Wrappedafterbreak}{\char`\}}}% 
            \def\PYGZca{\discretionary{\char`\^}{\Wrappedafterbreak}{\char`\^}}% 
            \def\PYGZam{\discretionary{\char`\&}{\Wrappedafterbreak}{\char`\&}}% 
            \def\PYGZlt{\discretionary{}{\Wrappedafterbreak\char`\<}{\char`\<}}% 
            \def\PYGZgt{\discretionary{\char`\>}{\Wrappedafterbreak}{\char`\>}}% 
            \def\PYGZsh{\discretionary{}{\Wrappedafterbreak\char`\#}{\char`\#}}% 
            \def\PYGZpc{\discretionary{}{\Wrappedafterbreak\char`\%}{\char`\%}}% 
            \def\PYGZdl{\discretionary{}{\Wrappedafterbreak\char`\$}{\char`\$}}% 
            \def\PYGZhy{\discretionary{\char`\-}{\Wrappedafterbreak}{\char`\-}}% 
            \def\PYGZsq{\discretionary{}{\Wrappedafterbreak\textquotesingle}{\textquotesingle}}% 
            \def\PYGZdq{\discretionary{}{\Wrappedafterbreak\char`\"}{\char`\"}}% 
            \def\PYGZti{\discretionary{\char`\~}{\Wrappedafterbreak}{\char`\~}}% 
        } 
        % Some characters . , ; ? ! / are not pygmentized. 
        % This macro makes them "active" and they will insert potential linebreaks 
        \newcommand*\Wrappedbreaksatpunct {% 
            \lccode`\~`\.\lowercase{\def~}{\discretionary{\hbox{\char`\.}}{\Wrappedafterbreak}{\hbox{\char`\.}}}% 
            \lccode`\~`\,\lowercase{\def~}{\discretionary{\hbox{\char`\,}}{\Wrappedafterbreak}{\hbox{\char`\,}}}% 
            \lccode`\~`\;\lowercase{\def~}{\discretionary{\hbox{\char`\;}}{\Wrappedafterbreak}{\hbox{\char`\;}}}% 
            \lccode`\~`\:\lowercase{\def~}{\discretionary{\hbox{\char`\:}}{\Wrappedafterbreak}{\hbox{\char`\:}}}% 
            \lccode`\~`\?\lowercase{\def~}{\discretionary{\hbox{\char`\?}}{\Wrappedafterbreak}{\hbox{\char`\?}}}% 
            \lccode`\~`\!\lowercase{\def~}{\discretionary{\hbox{\char`\!}}{\Wrappedafterbreak}{\hbox{\char`\!}}}% 
            \lccode`\~`\/\lowercase{\def~}{\discretionary{\hbox{\char`\/}}{\Wrappedafterbreak}{\hbox{\char`\/}}}% 
            \catcode`\.\active
            \catcode`\,\active 
            \catcode`\;\active
            \catcode`\:\active
            \catcode`\?\active
            \catcode`\!\active
            \catcode`\/\active 
            \lccode`\~`\~ 	
        }
    \makeatother

    \let\OriginalVerbatim=\Verbatim
    \makeatletter
    \renewcommand{\Verbatim}[1][1]{%
        %\parskip\z@skip
        \sbox\Wrappedcontinuationbox {\Wrappedcontinuationsymbol}%
        \sbox\Wrappedvisiblespacebox {\FV@SetupFont\Wrappedvisiblespace}%
        \def\FancyVerbFormatLine ##1{\hsize\linewidth
            \vtop{\raggedright\hyphenpenalty\z@\exhyphenpenalty\z@
                \doublehyphendemerits\z@\finalhyphendemerits\z@
                \strut ##1\strut}%
        }%
        % If the linebreak is at a space, the latter will be displayed as visible
        % space at end of first line, and a continuation symbol starts next line.
        % Stretch/shrink are however usually zero for typewriter font.
        \def\FV@Space {%
            \nobreak\hskip\z@ plus\fontdimen3\font minus\fontdimen4\font
            \discretionary{\copy\Wrappedvisiblespacebox}{\Wrappedafterbreak}
            {\kern\fontdimen2\font}%
        }%
        
        % Allow breaks at special characters using \PYG... macros.
        \Wrappedbreaksatspecials
        % Breaks at punctuation characters . , ; ? ! and / need catcode=\active 	
        \OriginalVerbatim[#1,codes*=\Wrappedbreaksatpunct]%
    }
    \makeatother

    % Exact colors from NB
    \definecolor{incolor}{HTML}{303F9F}
    \definecolor{outcolor}{HTML}{D84315}
    \definecolor{cellborder}{HTML}{CFCFCF}
    \definecolor{cellbackground}{HTML}{F7F7F7}
    
    % prompt
    \makeatletter
    \newcommand{\boxspacing}{\kern\kvtcb@left@rule\kern\kvtcb@boxsep}
    \makeatother
    \newcommand{\prompt}[4]{
        \ttfamily\llap{{\color{#2}[#3]:\hspace{3pt}#4}}\vspace{-\baselineskip}
    }
    

    
    % Prevent overflowing lines due to hard-to-break entities
    \sloppy 
    % Setup hyperref package
    \hypersetup{
      breaklinks=true,  % so long urls are correctly broken across lines
      colorlinks=true,
      urlcolor=urlcolor,
      linkcolor=linkcolor,
      citecolor=citecolor,
      }
    % Slightly bigger margins than the latex defaults
    
    \geometry{verbose,tmargin=1in,bmargin=1in,lmargin=1in,rmargin=1in}
    
    

\begin{document}
    
    \maketitle
    
    

    
    \hypertarget{implementation-of-linear-regression-using-gradient-descent}{%
\section{Implementation of linear regression using gradient
descent}\label{implementation-of-linear-regression-using-gradient-descent}}

    \begin{tcolorbox}[breakable, size=fbox, boxrule=1pt, pad at break*=1mm,colback=cellbackground, colframe=cellborder]
\prompt{In}{incolor}{9}{\boxspacing}
\begin{Verbatim}[commandchars=\\\{\}]
\PY{k+kn}{import} \PY{n+nn}{numpy} \PY{k}{as} \PY{n+nn}{np}

\PY{k}{class} \PY{n+nc}{linear\PYZus{}regression}\PY{p}{:}
    \PY{k}{def} \PY{n+nf}{\PYZus{}\PYZus{}init\PYZus{}\PYZus{}}\PY{p}{(}\PY{n+nb+bp}{self}\PY{p}{,} \PY{n}{learning\PYZus{}rate}\PY{p}{,} \PY{n}{iterations}\PY{p}{,} 
               \PY{n}{fit\PYZus{}intercept}\PY{o}{=}\PY{k+kc}{True}\PY{p}{,} \PY{n}{normalize}\PY{o}{=}\PY{k+kc}{False}\PY{p}{,} \PY{n}{coef}\PY{o}{=}\PY{k+kc}{None}\PY{p}{)}\PY{p}{:}
        \PY{n+nb+bp}{self}\PY{o}{.}\PY{n}{fit\PYZus{}intercept} \PY{o}{=} \PY{n}{fit\PYZus{}intercept}
        \PY{n+nb+bp}{self}\PY{o}{.}\PY{n}{normalize} \PY{o}{=} \PY{n}{normalize}
        \PY{n+nb+bp}{self}\PY{o}{.}\PY{n}{learning\PYZus{}rate} \PY{o}{=} \PY{n}{learning\PYZus{}rate}
        \PY{n+nb+bp}{self}\PY{o}{.}\PY{n}{iterations} \PY{o}{=} \PY{n}{iterations}
        \PY{n+nb+bp}{self}\PY{o}{.}\PY{n}{coef} \PY{o}{=} \PY{n}{coef}
        \PY{n+nb+bp}{self}\PY{o}{.}\PY{n}{cost} \PY{o}{=}\PY{l+m+mi}{0}
  
    \PY{c+c1}{\PYZsh{}Normalizing the X values by subracting it with the mean and dividing it with standard deviation.}
    \PY{k}{def} \PY{n+nf}{normalize\PYZus{}Data}\PY{p}{(}\PY{n+nb+bp}{self}\PY{p}{,}\PY{n}{X}\PY{p}{)}\PY{p}{:}
        \PY{n}{no\PYZus{}of\PYZus{}features}\PY{o}{=}\PY{n}{X}\PY{o}{.}\PY{n}{shape}\PY{p}{[}\PY{l+m+mi}{1}\PY{p}{]}
        \PY{n}{X\PYZus{}normalized}\PY{o}{=}\PY{n}{X}
    
        \PY{n}{Mean}\PY{o}{=}\PY{n}{np}\PY{o}{.}\PY{n}{zeros}\PY{p}{(}\PY{n}{no\PYZus{}of\PYZus{}features}\PY{p}{)}
        \PY{n}{Standard\PYZus{}deviation}\PY{o}{=}\PY{n}{np}\PY{o}{.}\PY{n}{zeros}\PY{p}{(}\PY{n}{no\PYZus{}of\PYZus{}features}\PY{p}{)}
        
        \PY{n}{X\PYZus{}normalized}\PY{o}{=} \PY{n}{X}\PY{o}{\PYZhy{}}\PY{n}{np}\PY{o}{.}\PY{n}{mean}\PY{p}{(}\PY{n}{X}\PY{p}{,}\PY{n}{axis}\PY{o}{=}\PY{l+m+mi}{0}\PY{p}{)}\PY{o}{/}\PY{n}{np}\PY{o}{.}\PY{n}{std}\PY{p}{(}\PY{n}{X}\PY{p}{,}\PY{n}{axis}\PY{o}{=}\PY{l+m+mi}{0}\PY{p}{)}
        \PY{k}{return} \PY{n}{X\PYZus{}normalized} 

 
    \PY{k}{def} \PY{n+nf}{fit}\PY{p}{(}\PY{n+nb+bp}{self}\PY{p}{,} \PY{n}{X}\PY{p}{,} \PY{n}{y}\PY{p}{)}\PY{p}{:}
       
    \PY{c+c1}{\PYZsh{}If we put normalize as true then only it will normalize the X values }
        \PY{k}{if} \PY{n+nb+bp}{self}\PY{o}{.}\PY{n}{normalize}\PY{p}{:}
            \PY{n}{X}\PY{o}{=}\PY{n+nb+bp}{self}\PY{o}{.}\PY{n}{normalize\PYZus{}Data}\PY{p}{(}\PY{n}{X}\PY{p}{)}
            
    \PY{c+c1}{\PYZsh{}Taking into account the number of columns and rows of the data and also the lenght of them.}
        \PY{n}{No\PYZus{}of\PYZus{}columns}\PY{o}{=}\PY{n}{X}\PY{o}{.}\PY{n}{shape}\PY{p}{[}\PY{l+m+mi}{1}\PY{p}{]}
        \PY{n}{no\PYZus{}of\PYZus{}rows}\PY{o}{=}\PY{n}{X}\PY{o}{.}\PY{n}{shape}\PY{p}{[}\PY{l+m+mi}{0}\PY{p}{]}
        \PY{n}{length\PYZus{}of\PYZus{}X}\PY{o}{=} \PY{n+nb}{len}\PY{p}{(}\PY{n}{X}\PY{p}{)}
        \PY{n}{length\PYZus{}of\PYZus{}y}\PY{o}{=} \PY{n+nb}{len}\PY{p}{(}\PY{n}{y}\PY{p}{)}
       
    \PY{c+c1}{\PYZsh{}If we put intercept to true then add one\PYZsq{}s column wise; c\PYZus{} adds values column wise }
    \PY{c+c1}{\PYZsh{}if we put intercept is false then the X remains the same}
        \PY{k}{if} \PY{n+nb+bp}{self}\PY{o}{.}\PY{n}{fit\PYZus{}intercept}\PY{p}{:}
            \PY{n}{Weights\PYZus{}dimension}\PY{o}{=}\PY{n}{No\PYZus{}of\PYZus{}columns} \PY{o}{+} \PY{l+m+mi}{1}
            \PY{n}{Modified\PYZus{}X}\PY{o}{=} \PY{n}{np}\PY{o}{.}\PY{n}{c\PYZus{}}\PY{p}{[}\PY{n}{np}\PY{o}{.}\PY{n}{ones}\PY{p}{(}\PY{p}{(}\PY{n}{length\PYZus{}of\PYZus{}X}\PY{p}{,}\PY{l+m+mi}{1}\PY{p}{)}\PY{p}{)}\PY{p}{,}\PY{n}{X}\PY{p}{]}
        \PY{k}{else}\PY{p}{:}
            \PY{n}{Weights\PYZus{}dimension}\PY{o}{=}\PY{n}{No\PYZus{}of\PYZus{}columns}
            \PY{n}{Modified\PYZus{}X} \PY{o}{=} \PY{n}{X}
            
        \PY{c+c1}{\PYZsh{}M is the weight vector. We are initializing the weight vector by taking zeros. We can also take random values to intialize it.}
        \PY{n+nb+bp}{self}\PY{o}{.}\PY{n}{M}\PY{o}{=}\PY{n}{np}\PY{o}{.}\PY{n}{zeros}\PY{p}{(}\PY{n}{Weights\PYZus{}dimension}\PY{p}{)}
        \PY{n}{X\PYZus{}T}\PY{o}{=}\PY{n}{np}\PY{o}{.}\PY{n}{transpose}\PY{p}{(}\PY{n}{Modified\PYZus{}X}\PY{p}{)}
        \PY{n+nb+bp}{self}\PY{o}{.}\PY{n}{cost}\PY{o}{=}\PY{l+m+mi}{0}
       
        
        \PY{k}{for} \PY{n}{i} \PY{o+ow}{in} \PY{n+nb}{range}\PY{p}{(}\PY{n+nb+bp}{self}\PY{o}{.}\PY{n}{iterations}\PY{p}{)}\PY{p}{:}
            \PY{c+c1}{\PYZsh{}y\PYZus{}hat is the y we predicted which can be obtained by multipying our X and M which is our weight vector}
            \PY{n}{y\PYZus{}hat}\PY{o}{=}\PY{n}{np}\PY{o}{.}\PY{n}{dot}\PY{p}{(}\PY{n}{Modified\PYZus{}X}\PY{p}{,}\PY{n+nb+bp}{self}\PY{o}{.}\PY{n}{M}\PY{p}{)}
            \PY{n}{error\PYZus{}vector}\PY{o}{=} \PY{n}{np}\PY{o}{.}\PY{n}{dot}\PY{p}{(}\PY{n}{X\PYZus{}T}\PY{p}{,}\PY{n}{y\PYZus{}hat}\PY{o}{\PYZhy{}}\PY{n}{y}\PY{p}{)}
            \PY{c+c1}{\PYZsh{}implementing the actual formulae weight=weight\PYZhy{}(1/number of rows)*learning rate*(summ of ypredicted\PYZhy{}actualy)*X}
            \PY{c+c1}{\PYZsh{}the error is the summation of ypred\PYZhy{}y actual }
            \PY{n+nb+bp}{self}\PY{o}{.}\PY{n}{M}\PY{o}{=}\PY{n+nb+bp}{self}\PY{o}{.}\PY{n}{M}\PY{o}{\PYZhy{}}\PY{p}{(}\PY{l+m+mi}{1}\PY{o}{/}\PY{n}{no\PYZus{}of\PYZus{}rows}\PY{p}{)}\PY{o}{*}\PY{n+nb+bp}{self}\PY{o}{.}\PY{n}{learning\PYZus{}rate}\PY{o}{*}\PY{p}{(}\PY{n}{error\PYZus{}vector}\PY{p}{)}
            \PY{n+nb+bp}{self}\PY{o}{.}\PY{n}{cost}\PY{o}{=} \PY{n}{np}\PY{o}{.}\PY{n}{sum}\PY{p}{(}\PY{p}{(}\PY{n}{y\PYZus{}hat}\PY{o}{\PYZhy{}}\PY{n}{y}\PY{p}{)}\PY{o}{*}\PY{o}{*}\PY{l+m+mi}{2}\PY{p}{)}\PY{o}{/}\PY{l+m+mi}{2}\PY{o}{*}\PY{n}{no\PYZus{}of\PYZus{}rows}
        \PY{k}{return} \PY{n+nb+bp}{self}\PY{o}{.}\PY{n}{M}\PY{p}{,}\PY{n+nb+bp}{self}\PY{o}{.}\PY{n}{cost}
    
    


    \PY{k}{def} \PY{n+nf}{predict}\PY{p}{(}\PY{n+nb+bp}{self}\PY{p}{,} \PY{n}{X}\PY{p}{)}\PY{p}{:}
        \PY{n}{length\PYZus{}of\PYZus{}X}\PY{o}{=} \PY{n+nb}{len}\PY{p}{(}\PY{n}{X}\PY{p}{)}
    
        \PY{k}{if} \PY{n+nb+bp}{self}\PY{o}{.}\PY{n}{fit\PYZus{}intercept}\PY{p}{:}
            \PY{n}{Modified\PYZus{}X}\PY{o}{=} \PY{n}{np}\PY{o}{.}\PY{n}{c\PYZus{}}\PY{p}{[}\PY{n}{np}\PY{o}{.}\PY{n}{ones}\PY{p}{(}\PY{p}{(}\PY{n}{length\PYZus{}of\PYZus{}X}\PY{p}{,}\PY{l+m+mi}{1}\PY{p}{)}\PY{p}{)}\PY{p}{,}\PY{n}{X}\PY{p}{]}
        \PY{k}{else}\PY{p}{:}
            \PY{n}{Modified\PYZus{}X} \PY{o}{=} \PY{n}{X}
            
            \PY{c+c1}{\PYZsh{}The predicted value }
            \PY{n}{prediction}\PY{o}{=}\PY{n}{np}\PY{o}{.}\PY{n}{dot}\PY{p}{(}\PY{n}{X}\PY{p}{,}\PY{n+nb+bp}{self}\PY{o}{.}\PY{n}{M}\PY{p}{)}
        \PY{k}{return} \PY{n}{prediction}
\end{Verbatim}
\end{tcolorbox}

    \hypertarget{problem-1.2-10-points}{%
\subsection{Problem 1.2 (10 points)}\label{problem-1.2-10-points}}

\begin{itemize}
\tightlist
\item
  Split the Boston Housing dataset into train and test sets (70\% and
  30\%, respectively) (5 points).
\item
  Fit your linear regression implementation using the training set and
  print your model's coefficients. Make predictions for the test set
  using your fitted model (5 points).
\end{itemize}

    \begin{tcolorbox}[breakable, size=fbox, boxrule=1pt, pad at break*=1mm,colback=cellbackground, colframe=cellborder]
\prompt{In}{incolor}{53}{\boxspacing}
\begin{Verbatim}[commandchars=\\\{\}]
\PY{k+kn}{from} \PY{n+nn}{sklearn}\PY{n+nn}{.}\PY{n+nn}{datasets} \PY{k}{import} \PY{n}{load\PYZus{}boston}
\PY{k+kn}{from} \PY{n+nn}{sklearn}\PY{n+nn}{.}\PY{n+nn}{model\PYZus{}selection} \PY{k}{import} \PY{n}{train\PYZus{}test\PYZus{}split}

\PY{c+c1}{\PYZsh{}loading the boston data set }
\PY{n}{dataset} \PY{o}{=} \PY{n}{load\PYZus{}boston}\PY{p}{(}\PY{p}{)}
\PY{n}{X} \PY{o}{=} \PY{n}{dataset}\PY{o}{.}\PY{n}{data}
\PY{n}{y} \PY{o}{=} \PY{n}{dataset}\PY{o}{.}\PY{n}{target}

\PY{c+c1}{\PYZsh{}Splitting the data set into 70percent train and 30percent test }
\PY{n}{X\PYZus{}train}\PY{p}{,}\PY{n}{X\PYZus{}test}\PY{p}{,}\PY{n}{y\PYZus{}train}\PY{p}{,}\PY{n}{y\PYZus{}test}\PY{o}{=}\PY{n}{train\PYZus{}test\PYZus{}split}\PY{p}{(}\PY{n}{X}\PY{p}{,}\PY{n}{y}\PY{p}{,}\PY{n}{test\PYZus{}size}\PY{o}{=}\PY{l+m+mf}{0.3}\PY{p}{)}

\PY{c+c1}{\PYZsh{}Calling the linear regression class which i have designed above}
\PY{n}{regresser}\PY{o}{=}\PY{n}{linear\PYZus{}regression}\PY{p}{(}\PY{n}{learning\PYZus{}rate}\PY{o}{=}\PY{l+m+mf}{0.000001}\PY{p}{,}\PY{n}{iterations}\PY{o}{=}\PY{l+m+mi}{5000}\PY{p}{,}\PY{n}{fit\PYZus{}intercept}\PY{o}{=}\PY{k+kc}{False}\PY{p}{,} \PY{n}{normalize}\PY{o}{=}\PY{k+kc}{True}\PY{p}{,} \PY{n}{coef}\PY{o}{=}\PY{k+kc}{None}\PY{p}{)}

\PY{c+c1}{\PYZsh{}fitting the model }
\PY{n}{regresser}\PY{o}{.}\PY{n}{fit}\PY{p}{(}\PY{n}{X\PYZus{}train}\PY{p}{,}\PY{n}{y\PYZus{}train}\PY{p}{)}

\PY{c+c1}{\PYZsh{}predicted values }
\PY{n}{y\PYZus{}pred}\PY{o}{=}\PY{n}{regresser}\PY{o}{.}\PY{n}{predict}\PY{p}{(}\PY{n}{X\PYZus{}test}\PY{p}{)}
\PY{n}{y\PYZus{}pred}
\end{Verbatim}
\end{tcolorbox}

            \begin{tcolorbox}[breakable, size=fbox, boxrule=.5pt, pad at break*=1mm, opacityfill=0]
\prompt{Out}{outcolor}{53}{\boxspacing}
\begin{Verbatim}[commandchars=\\\{\}]
array([22.05022161, 20.52910395, 21.64194652, 24.02955968, 21.74400224,
        6.55804271, 27.64541769, 21.40999715, 22.20333231, 20.92341822,
       22.78100979, 31.07811375, 23.53969798, 24.33393075, 20.79586313,
       21.5151255 , 20.22217616,  6.74661551, 23.42710707, 18.30922938,
       32.16900166, 23.54177321, 22.33651012, 18.7086258 , 21.5246299 ,
       21.77785827, 21.91653255, 22.74537039, 23.0071897 , 22.54171012,
       20.23008668, 21.53231062, 22.65394127, 21.23086582, 23.92653525,
       23.68685044, 21.44488765, 21.00939822, 22.49707866, 23.9833305 ,
       19.80955846, 23.23436628, 23.04043995, 21.67085787, 16.59595134,
       22.02292817, 21.91530143, 22.25810157, 20.91945964, 23.41097142,
       22.42321506, 22.28111958, 17.37279039, 22.65224749, 26.2221944 ,
       22.35096623, 24.59564973, 29.12692301, 22.23082348, 22.85632858,
       22.81963338, 28.09093858, 20.33912184, 21.73392885, 21.32218863,
       20.37833627, 22.96939231, 21.80390328, 22.90076625, 19.98205489,
       24.32859461, 32.93501568, 21.44633107, 25.06578929, 23.40735111,
       21.04081823, 25.41266274, 24.75679384, 24.87875843, 22.50384639,
       22.5922146 , 23.84327067, 22.25518405, 23.28109463, 24.15131417,
       22.1856263 , 21.42241906, 21.62893799, 22.55862451, 23.46405785,
        8.56867858, 28.61155161, 25.37539995, 32.52748607, 20.71858835,
       21.87599746, 22.32688155, 28.96870199, 28.47691431, 23.14034397,
       22.69919227, 33.58816496, 24.10722882, 22.88321656, 24.87396515,
       22.23726193, 20.22838377, 24.09533792, 22.60182273, 31.85852659,
       21.51596963, 20.97612627, 24.19041001,  5.8829942 , 22.56980762,
       24.00227093, 22.29643755, 22.47729961, 21.06562251, 18.63803101,
       19.69199033, 21.7029517 , 31.93703859, 24.51776124, 24.90428517,
       20.95484523,  2.8606782 , 23.26997388,  4.92153222, 22.6675151 ,
       19.94150005,  1.13020492, 19.04573509, 21.9345138 , 21.11782673,
       21.65260144, 26.74534735, 22.18616945, 22.82073554,  2.49719509,
       22.37968056, 21.90424662, 31.36918681, 21.85576578, 21.72179145,
       31.39445969, 22.22565913, 23.49279076, 22.70259458, 23.02056128,
       23.30073871, 21.3434654 ])
\end{Verbatim}
\end{tcolorbox}
        
    \hypertarget{questions}{%
\subsection{Questions}\label{questions}}

\begin{enumerate}
\def\labelenumi{\arabic{enumi}.}
\tightlist
\item
  How do you interpret that a variable causes a model's mean square
  error to increase?
\end{enumerate}

\begin{itemize}
\tightlist
\item
  Answer: So basically we plot the best fit line for our linear
  regression model and check the number of data points which are close
  to the best fit line.Generally the term used to check the closeness is
  Rsquare. As we check the number of points close to the best fit line
  there may be some data points which are very far away from the line
  these are nothing but outliers. The outliers are the points which
  cause our models mean square error to increase.
\end{itemize}

\begin{enumerate}
\def\labelenumi{\arabic{enumi}.}
\setcounter{enumi}{1}
\tightlist
\item
  Why we would want to normalize our variables?
\end{enumerate}

\begin{itemize}
\tightlist
\item
  Answer: In some cases few points in our data will have very high
  values which will dominate the other data points which will lead to a
  confused bad model. Instead if we normalize the data we wont be facing
  this domination issue. Usually we apply log function for normalization
\end{itemize}

\begin{enumerate}
\def\labelenumi{\arabic{enumi}.}
\setcounter{enumi}{2}
\tightlist
\item
  A model fitted using the exact same split dataset with normalized
  values will generate the same coefficients as a model that was fitted
  using values that haven't been normalized. Clearly state whether that
  statement is true or false and explain your reasoning.
\end{enumerate}

\begin{itemize}
\tightlist
\item
  Answer: False reason:- if we dont normalize there may be some points
  which may show domination which may influence the model.So by
  normalizing we wont have many dominant points.So as we are normalizing
  the coeffcients will be differnt.On the other side if the data is
  already normalized then coefficients wont change.
\end{itemize}

    \begin{tcolorbox}[breakable, size=fbox, boxrule=1pt, pad at break*=1mm,colback=cellbackground, colframe=cellborder]
\prompt{In}{incolor}{ }{\boxspacing}
\begin{Verbatim}[commandchars=\\\{\}]

\end{Verbatim}
\end{tcolorbox}


    % Add a bibliography block to the postdoc
    
    
    
\end{document}
